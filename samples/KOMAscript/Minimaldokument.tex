\documentclass{scrreprt}

\usepackage[ngerman]{babel}		% Sprachen
\usepackage[latin1]{inputenc} % Eingabe von �,�,�,� erlaubt 
\usepackage[T1]{fontenc}			% => Umlauten koennen richtig getrennt werden

\begin{document}

\chapter{Kapitel 1}
%\label{sec:Kapitel1}

\section{�berschrift 1.1}
\label{sec:�berschrift11}
Text 1.1 Weit hinten, hinter den Wortbergen, fern der L�nder Vokalien und Konsonantien leben die Blindtexte. Abgeschieden wohnen Sie in Buchstabhausen an der K�ste des Semantik, eines gro�en Sprachozeans. Ein kleines B�chlein namens Duden flie�t durch ihren Ort und versorgt sie mit den n�tigen Regelialien. 
Es ist ein paradiesmatisches Land, in dem einem gebratene Satzteile in den Mund fliegen. Nicht einmal von der allm�chtigen Interpunktion werden die Blindtexte beherrscht - ein geradezu unorthographisches Leben. Eines Tages aber beschloss eine kleine Zeile Blindtext, ihr Name war Lorem Ipsum, hinaus zu gehen in die weite Grammatik.


\subsection{Unter�berschrift 1.1.1}
%\label{sec:�berschrift111}
Text 1.1.1. Der gro�e Oxmox riet ihr davon ab, da es dort wimmele von b�sen Kommata, wilden Fragezeichen und hinterh�ltigen Semikoli, doch das Blindtextchen lie� sich nicht beirren. Es packte seine sieben Versalien, schob sich sein Initial in den G�rtel und machte sich auf den Weg. Als es die ersten H�gel des Kursivgebirges erklommen hatte, warf es einen letzten Blick zur�ck auf die Skyline seiner Heimatstadt Buchstabhausen, die Headline von Alphabetdorf und die Subline seiner eigenen Stra�e, der Zeilengasse. Wehm�tig lief ihm eine rhetorische Frage �ber die Wange, dann setzte es seinen Weg fort.

\section{�berschrift 2}
%\label{sec:�berschrift2}
Unterwegs traf es eine Copy. Die Copy warnte das Blindtextchen, da, wo sie herk�me w�re sie zigmal umgeschrieben worden und alles, was von ihrem Ursprung noch �brig w�re, sei das Wort "und" und das Blindtextchen solle umkehren und wieder in sein eigenes, sicheres Land zur�ckkehren. Doch alles Gutzureden konnte es nicht �berzeugen und so dauerte es nicht lange, bis ihm ein paar heimt�ckische Werbetexter auflauerten, es mit Longe und Parole betrunken machten und es dann in ihre Agentur schleppten, wo sie es f�r ihre Projekte wieder und wieder missbrauchten. Und wenn es nicht umgeschrieben wurde, dann benutzen Sie es immer noch.


\section{�berschrift 3}
%\label{sec:�berschrift3}
Text 3 Weit hinten, hinter den Wortbergen, fern der L�nder Vokalien und Konsonantien leben die Blindtexte. Abgeschieden wohnen Sie in Buchstabhausen an der K�ste des Semantik, eines gro�en Sprachozeans. Ein kleines B�chlein namens Duden flie�t durch ihren Ort und versorgt sie mit den n�tigen Regelialien. Es ist ein paradiesmatisches Land, in dem einem gebratene Satzteile in den Mund fliegen. Nicht einmal von der allm�chtigen Interpunktion werden die Blindtexte beherrscht - ein geradezu unorthographisches Leben. Eines Tages aber beschloss eine kleine Zeile Blindtext, ihr Name war Lorem Ipsum, hinaus zu gehen in die weite Grammatik.

\subsection{Unterabschnitt}
%\label{sec:Unterabschnitt}
Text Text
\subsubsection{Unter-Unterabschnitt}
%\label{sec:UnterUnterabschnitt}
Text Text
\paragraph{Ein Absatz}
%\label{sec:Absatz}
Text Text
\subparagraph{Unterabsatz}
%\label{sec:Unterabsatz}
Text Text

\end{document} 