\documentclass[%
	12pt,%11pt, 10pt%
	a4paper,%a5paper, letterpaper, legalpaper, executivepaper
%	onelinecaption,% noonelinecaption,%
	oneside,%twoside,%
%	onecolumn,%twocolumn,%
% openany, openright,%
% cleardoublestandard,% cleardoubleplain, %cleardoubleempty
% notitlepage,%titlepage,%	
 liststotoc, idxtotoc, bibtotoc, %bibtotocnumbered, liststotocnumbered
% tocindent,% tocleft,%
% listsindent,% listsleft%
	halfparskip,% parindent, parskip, parskip*, parskip+, parskip, halfparskip*, halfparskip+, halfparskip,
	nochapterprefix,%chapterprefix,%
	appendixprefix, %noappendixprefix,%
%	headsepline,headnosepline%
%	footsepline,footnosepline%
% bigheadings, normalheadings, 
smallheadings,%
%%Formatierungsparameter
%	abstractoff,% abstracton%
% pointednumbers,% pointlessnumbers%
% leqno, fleqn%
%	tablecaptionbelow,% tablecaptionbelow%
% origlongtable,%
% openbib,%
% final, draft%
]{scrreprt}
%]{scrbook}
%]{scrartcl}
%]{scrlttr2}
%
%]{article}
%]{report}
%]{letter}
%]{book}
%]{slides}

\usepackage{bibgerm}					% deutsche Bibliography
\usepackage[ngerman]{babel}		% Sprachen
\usepackage[latin1]{inputenc} % Eingabe von �,�,�,� erlaubt 
\usepackage[T1]{fontenc}			% Umlauten koennen richtig getrennt werden

\usepackage{lmodern}

%======================================================================
%   Darstellung des Glossars und des Indexes
%======================================================================
%\usepackage{makeidx}
\usepackage{booktabs}
%\usepackage{caption}							% mehrzeilige Captions ausrichten
%\captionsetup{format=hang,margin=5pt,labelfont=bf,position=bottom,aboveskip=1em,belowskip=1em}
%%\setcapindent{-1em}							 	% Zeilenumbruch bei Bildbeschreibungen.
%\setlength{\belowcaptionskip}{1em}

%\usepackage{multicol}            	% mehrere Spalten

%\usepackage{setspace}            % Zeilenabstand einstellbar
%\onehalfspacing                  % eineinhalbzeilig einstellen

%\usepackage{theorem}
%\usepackage{thmbox}
%\usepackage{amsthm}


%\theoremstyle{break}
\newtheorem{definition}{Definition}

\newtheorem{thm}{Theorem}[chapter]
\newtheorem{cor}[thm]{Corollary}
\newtheorem{prop}{Proposition}
\newtheorem{lem}[thm]{Lemma}
\newtheorem{proof}{Proof}

%\theoremstyle{remark}
%\newtheorem*{rmk}{Remark}
\newtheorem{rmk}{Remark}

%\theoremstyle{plain}
%\newtheorem*{Ahlfors}{Ahlfors' Lemma}
\newtheorem{Ahlfors}{Ahlfors' Lemma}

%\newtheoremstyle{note}% name
%  {3pt}%      Space above
%  {3pt}%      Space below
%  {}%         Body font
%  {}%         Indent amount (empty = no indent, \parindent = para indent)
%  {\itshape}% Thm head font
%  {:}%        Punctuation after thm head
%  {.5em}%     Space after thm head: " " = normal interword space;
%        %       \newline = linebreak
%  {}%         Thm head spec (can be left empty, meaning `normal')
%
%\theoremstyle{note}
\newtheorem{note}{Note}

%\newtheoremstyle{citing}% name
%  {3pt}%      Space above, empty = `usual value'
%  {3pt}%      Space below
%  {\itshape}% Body font
%  {}%         Indent amount (empty = no indent, \parindent = para indent)
%  {\bfseries}% Thm head font
%  {.}%        Punctuation after thm head
%  {.5em}%     Space after thm head: " " = normal interword space;
%        %       \newline = linebreak
%  {\thmnote{#3}}% Thm head spec
%
%\theoremstyle{citing}
%\newtheorem*{varthm}{}% all text supplied in the note
\newtheorem{varthm}{}% all text supplied in the note

%\newtheoremstyle{break}% name
%  {9pt}%      Space above, empty = `usual value'
%  {9pt}%      Space below
%  {\itshape}% Body font
%  {}%         Indent amount (empty = no indent, \parindent = para indent)
%  {\bfseries}% Thm head font
%  {.}%        Punctuation after thm head
%  {\newline}% Space after thm head: \newline = linebreak
%  {}%         Thm head spec
%
%\theoremstyle{break}
\newtheorem{bthm}{B-Theorem}

%\theoremstyle{exercise}
\newtheorem{exer}{Exercise}

%\swapnumbers
%\theoremstyle{plain}
\newtheorem{thmsw}{Theorem}[chapter]
\newtheorem{corsw}[thmsw]{Corollary}
\newtheorem{propsw}{Proposition}
\newtheorem{lemsw}[thmsw]{Lemma}

%%    Because the amsmath pkg is not used, we need to define a couple of
%%    commands in more primitive terms.
\let\lvert=|\let\rvert=|
\newcommand{\Ric}{\mathop{\mathrm{Ric}}\nolimits}
%
%%    Dispel annoying problem of slightly overlong lines:
%\addtolength{\textwidth}{8pt}

%\usepackage{varioref}%

%\usepackage{array}
%\usepackage{tabularx}
%\usepackage{longtable}

%-------------------------------Farbdefinitionen------------------------------------------------------
\usepackage{color}
\definecolor{sun1}{rgb}{0.2,0.2,0.4} %dunkelblau
\definecolor{sun2}{rgb}{0.4,0.4,0.6} %mittel
\definecolor{sun3}{rgb}{0.6,0.6,0.8} %mittel
\definecolor{sun4}{rgb}{0.8,0.8,1.0} %hellblau
\definecolor{sun5}{rgb}{0.9,0.9,1.0} %hellblau
\definecolor{black}{rgb}{0,0,0} %hellblau

%-------------------------------Kopf- und Fu�zeile----------------------------------------------------
%\usepackage{scrpage2}
%\pagestyle{scrheadings}
%\pagestyle{scrplain}

%\ihead[scrplain-innen]{scrheadings-innen}
%\chead[scrplain-zentriert]{scrheadings-zentriert}
%\ohead[scrplain-au�en]{scrheadings-au�en}
%\ifoot[scrplain-innen]{scrheadings-innen}
%\cfoot[scrplain-zentriert]{scrheadings-zentriert}
%\ofoot[scrplain-au�en]{scrheadings-au�en}
%
%\usepackage{fancyhdr}
%\pagestyle{fancy}
%\addtolength{\headheight}{\baselineskip}
	
%\fancyhf{}   % l�scht alle Felder
%
%\fancyhead[RO]{\nouppercase{\sffamily{\rightmark}}}%Kopfzeile rechts bzw. au�en
%\fancyhead[LE]{\nouppercase{\sffamily{\leftmark}}}
%\renewcommand{\headrule}{\vbox to 0pt{\hbox to \headwidth{\dotfill}\vss}} %gepunktete Linie
%
%\fancyfoot[CO,CE]{\color{sun1}\sffamily{\thepage}}					%Fu�zeile rechts bzw. au�en
%\renewcommand{\footrule}{\vbox to 0pt{\hbox to \headwidth{\dotfill}\vss}} %gepunktete Linie
%
%%extra Formatierungen der Kopf- und Fu�zeilen f�r Kapitelseiten
%\fancypagestyle{plain}{
%	\fancyhf{}   % l�scht alle Felder
%	
%	\fancyhead[LE,RO]{\nouppercase{\sffamily{\leftmark}}}%Kopfzeile rechts bzw. au�en
%	\renewcommand{\headrule}{\vbox to 0pt{\hbox to \headwidth{\dotfill}\vss}} %gepunktete Linie
% 	
%	\fancyfoot[CO,CE]{\color{sun1}\sffamily{\thepage}}					%Fu�zeile rechts bzw. au�en
%	\renewcommand{\footrule}{\vbox to 0pt{\hbox to \headwidth{\dotfill}\vss}} %gepunktete Linie
%}

%======================================================================
%	Erkennen des Bearbeitungsmodus f�r weitere Paketoptionen
%======================================================================
\RequirePackage{ifthen}

\newif\ifhtm 
\ifx \HCode\undefined \htmfalse \else	\htmtrue \fi

\newif\ifpdf
\ifx \pdfoutput\undefined 
	\pdffalse 		%latex in DVI mode
\else
	\ifthenelse{\number\pdfoutput<1}%
 	{\pdffalse} 	%pdflatex in DVI mode
 	{\pdftrue} 		%pdflatex in PDF mode
\fi

%======================================================================
%		PDF <=> DVI <=> XHTML 
%======================================================================
\ifhtm%
		% Bearbeitung mit tex4ht
		\usepackage{graphicx}%
		\usepackage[tex4ht]{hyperref}%
\else%
	\ifpdf%
		%Bearbeitung mit pdftex im PDF-Modus	
		\usepackage[pdftex]{graphicx}%
		\usepackage[pdftex]{hyperref}%
		\pdfcompresslevel=0%						Damit wird die PDF-Quelldatei lesbar
		\pdfoptionpdfminorversion=6%		Bestimmt die PDF - Version der Ausgabe
	%\pdfadjustspacing=0%							0, 1 oder 2 �nderung nicht erkannt
%		\AtBeginDocument{\usepackage[tagged, flatstructure]{accessibility}}	% Makros in accessibility.sty 
		\AtBeginDocument{\usepackage[tagged, highstructure]{accessibility}}	% Makros in accessibility.sty 
	\else%
		%Bearbeitung mit latex oder pdftex im DVI-Modus	
		\usepackage{graphicx}%
		\usepackage{hyperref}%
		\newcommand{\alt}[1]{}%
		\newcommand{\thead}[1]{\textbf{#1}}
	\fi%
\fi%

% Einstellungen des Hyperref-Paketes
%\AtBeginDocument{%
	\hypersetup{%
%	unicode = true,%
	colorlinks=true,%
	linkcolor=sun1,%
	citecolor=sun1,%
	filecolor=sun1,%
	menucolor=sun1,%
	pagecolor=sun1,%
	urlcolor=sun1,%
%	pdftitle	= {\dctitle}, %
%	pdfsubject	= {\dcsubject, \dcdate}, %
%	pdfauthor	= {\dcauthorname~\dcauthorsurname, \dcauthoremail}, %
%	pdfkeywords	= {\dckeywords}, %
	pdfcreator	= {pdfTeX with Hyperref and Thumbpdf}, %
	pdfproducer	= {LaTeX, hyperref, thumbpdf}, %
	bookmarksnumbered,%
%	hypertexnames = true,%
	plainpages = true,%
	}%
%}
%\usepackage[hyper=true, style=altlist, toc=true]{glossary}
%\usepackage{glossary}

%======================================================================
%\makeindex
%\makeglossary
%%makeindex -s minimaldokument.ist -t minimaldokument.glg -o minimaldokument.gls minimaldokument.glo
%======================================================================
\begin{document}

%%%======================================================================================================
\chapter{theorem}


Es ist ein paradiesmatisches Land, in dem einem gebratene Satzteile in den Mund fliegen. Nicht einmal von der allm�chtigen Interpunktion werden die Blindtexte beherrscht - ein geradezu unorthographisches Leben. Eines Tages aber beschloss eine kleine Zeile Blindtext, ihr Name war Lorem Ipsum, hinaus zu gehen in die weite Grammatik.

\begin{definition}[Definition]
Ein Theorem vom Typ Definition.
\end{definition}

Es ist ein paradiesmatisches Land, in dem einem gebratene Satzteile in den Mund fliegen. Nicht einmal von der allm�chtigen Interpunktion werden die Blindtexte beherrscht - ein geradezu unorthographisches Leben. Eines Tages aber beschloss eine kleine Zeile Blindtext, ihr Name war Lorem Ipsum, hinaus zu gehen in die weite Grammatik.

\begin{definition}
Hallo ich bin eine Theorem vom Typ Definition.
\end{definition}

Es ist ein paradiesmatisches Land, in dem einem gebratene Satzteile in den Mund fliegen. Nicht einmal von der allm�chtigen Interpunktion werden die Blindtexte beherrscht - ein geradezu unorthographisches Leben. Eines Tages aber beschloss eine kleine Zeile Blindtext, ihr Name war Lorem Ipsum, hinaus zu gehen in die weite Grammatik.


\chapter{Test of standard theorem styles}

Ahlfors' Lemma gives the principal criterion for obtaining lower bounds
on the Kobayashi metric.

\begin{Ahlfors}
Let $ds^2 = h(z)\lvert dz\rvert^2$ be a Hermitian pseudo-metric on
$\mathbf{D}_r$, $h\in C^2(\mathbf{D}_r)$, with $\omega$ the associated
$(1,1)$-form. If $\Ric\omega\geq\omega$ on $\mathbf{D}_r$,
then $\omega\leq\omega_r$ on all of $\mathbf{D}_r$ (or equivalently,
$ds^2\leq ds_r^2$).
\end{Ahlfors}

\begin{lem}[negatively curved families]
Let $\{ds_1^2,\dots,ds_k^2\}$ be a negatively curved family of metrics
on $\mathbf{D}_r$, with associated forms $\omega^1$, \dots, $\omega^k$.
Then $\omega^i \leq\omega_r$ for all $i$.
\end{lem}

Then our main theorem:
\begin{thm}\label{pigspan}
Let $d_{\max}$ and $d_{\min}$ be the maximum, resp.\ minimum distance
between any two adjacent vertices of a quadrilateral $Q$. Let $\sigma$
be the diagonal pigspan of a pig $P$ with four legs.
Then $P$ is capable of standing on the corners of $Q$ iff
%\begin{equation}\label{sdq}
%\sigma\geq \sqrt{d_{\max}^2+d_{\min}^2}.
%\end{equation}
\end{thm}

\begin{cor}
Admitting reflection and rotation, a three-legged pig $P$ is capable of
standing on the corners of a triangle $T$ iff %(\ref{sdq}) holds.
\end{cor}

\begin{rmk}
As two-legged pigs generally fall over, the case of a polygon of order
$2$ is uninteresting.
\end{rmk}

\chapter{Custom theorem styles}

\begin{exer}
Generalize Theorem~%\ref{pigspan}
 to three and four dimensions.
\end{exer}

\begin{note}
This is a test of the custom theorem style `note'. It is supposed to have
variant fonts and other differences.
\end{note}

\begin{bthm}
Test of the `linebreak' style of theorem heading.
\end{bthm}

This is a test of a citing theorem to cite a theorem from some other source.

\begin{varthm}[Theorem 3.6 in %\cite{thatone}
]
No hyperlinking available here yet \dots\ but that's not a
bad idea for the future.
\end{varthm}

\chapter{The proof environment}

\begin{proof}
Here is a test of the proof environment.
\end{proof}

\begin{proof}[Proof of Theorem %\ref{pigspan}
]
And another test.
\end{proof}

\begin{proof}[Proof \textup(necessity\textup)]
And another.
\end{proof}

\begin{proof}[Proof \textup(sufficiency\textup)]
And another, ending with a display:
%\[
%1+1=2\,. 
%\]
\end{proof}

\chapter{Test of number-swapping}

This is a repeat of the first chapter but with numbers in theorem heads
swapped to the left.

Ahlfors' Lemma gives the principal criterion for obtaining lower bounds
on the Kobayashi metric.
\begin{Ahlfors}
Let $ds^2 = h(z)\lvert dz\rvert^2$ be a Hermitian pseudo-metric on
$\mathbf{D}_r$, $h\in C^2(\mathbf{D}_r)$, with $\omega$ the associated
$(1,1)$-form. If $\Ric\omega\geq\omega$ on $\mathbf{D}_r$,
then $\omega\leq\omega_r$ on all of $\mathbf{D}_r$ (or equivalently,
$ds^2\leq ds_r^2$).
\end{Ahlfors}

\begin{lemsw}[negatively curved families]
Let $\{ds_1^2,\dots,ds_k^2\}$ be a negatively curved family of metrics
on $\mathbf{D}_r$, with associated forms $\omega^1$, \dots, $\omega^k$.
Then $\omega^i \leq\omega_r$ for all $i$.
\end{lemsw}

Then our main theorem:
\begin{thmsw}
Let $d_{\max}$ and $d_{\min}$ be the maximum, resp.\ minimum distance
between any two adjacent vertices of a quadrilateral $Q$. Let $\sigma$
be the diagonal pigspan of a pig $P$ with four legs.
Then $P$ is capable of standing on the corners of $Q$ iff
%\begin{equation}\label{sdqsw}
%\sigma\geq \sqrt{d_{\max}^2+d_{\min}^2}.
%\end{equation}
\end{thmsw}

\begin{corsw}
Admitting reflection and rotation, a three-legged pig $P$ is capable of
standing on the corners of a triangle $T$ iff %(\ref{sdqsw}) holds.
\end{corsw}

\begin{thebibliography}{99}
\bibitem{thatone} Dummy entry.
\end{thebibliography}


\end{document}